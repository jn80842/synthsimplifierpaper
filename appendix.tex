%% For double-blind review submission, w/o CCS and ACM Reference (max submission space)
\documentclass[acmsmall,review,anonymous]{acmart}\settopmatter{printfolios=true,printccs=false,printacmref=false}
%% For double-blind review submission, w/ CCS and ACM Reference
%\documentclass[sigplan,review,anonymous]{acmart}\settopmatter{printfolios=true}
%% For single-blind review submission, w/o CCS and ACM Reference (max submission space)
%\documentclass[sigplan,review]{acmart}\settopmatter{printfolios=true,printccs=false,printacmref=false}
%% For single-blind review submission, w/ CCS and ACM Reference
%\documentclass[sigplan,review]{acmart}\settopmatter{printfolios=true}
%% For final camera-ready submission, w/ required CCS and ACM Reference
%\documentclass[sigplan]{acmart}\settopmatter{}
\usepackage{amsmath}
\usepackage{xspace}
\usepackage{natbib}
\usepackage{csvsimple}
\usepackage{microtype}
\usepackage{tikz}
\usepackage{tikz-qtree}
\usepackage{tabularx}
\usepackage{makecell}
\usetikzlibrary{hobby,arrows,backgrounds,calc,trees}

\pgfdeclarelayer{background}
\pgfsetlayers{background,main}

\newcommand{\convexpath}[2]{
[   
    create hullnodes/.code={
        \global\edef\namelist{#1}
        \foreach [count=\counter] \nodename in \namelist {
            \global\edef\numberofnodes{\counter}
            \node at (\nodename) [draw=none,name=hullnode\counter] {};
        }
        \node at (hullnode\numberofnodes) [name=hullnode0,draw=none] {};
        \pgfmathtruncatemacro\lastnumber{\numberofnodes+1}
        \node at (hullnode1) [name=hullnode\lastnumber,draw=none] {};
    },
    create hullnodes
]
($(hullnode1)!#2!-90:(hullnode0)$)
\foreach [
    evaluate=\currentnode as \previousnode using \currentnode-1,
    evaluate=\currentnode as \nextnode using \currentnode+1
    ] \currentnode in {1,...,\numberofnodes} {
  let
    \p1 = ($(hullnode\currentnode)!#2!-90:(hullnode\previousnode)$),
    \p2 = ($(hullnode\currentnode)!#2!90:(hullnode\nextnode)$),
    \p3 = ($(\p1) - (hullnode\currentnode)$),
    \n1 = {atan2(\y3,\x3)},
    \p4 = ($(\p2) - (hullnode\currentnode)$),
    \n2 = {atan2(\y4,\x4)},
    \n{delta} = {-Mod(\n1-\n2,360)}
  in 
    {-- (\p1) arc[start angle=\n1, delta angle=\n{delta}, radius=#2] -- (\p2)}
}
-- cycle
}
\tikzset{
  leaf/.style={sibling distance=5mm}
}

%\definecolor{uwpurple}{RGB}{128,0,128}
%\newcommand{\jln}[1]{\textcolor{uwpurple}{\textit{[{#1} --JLN]}}}
%\newcommand{\sak}[1]{\textcolor{olive}{\textit{[{#1} --SK]}}}
%\newcommand{\aba}[1]{\textcolor{red}{\textit{[{#1} --AA]}}}
\usepackage{syntax}
%% Conference information
%% Supplied to authors by publisher for camera-ready submission;
%% use defaults for review submission.
%\acmConference[PLDI'20]{ACM SIGPLAN Conference on Programming Language Design and Implementation}{June 2020}{London, UK}
%\acmYear{2020}
\acmISBN{} % \acmISBN{978-x-xxxx-xxxx-x/YY/MM}
\acmDOI{} % \acmDOI{10.1145/nnnnnnn.nnnnnnn}
\startPage{1}

%% Macros for quantities
\newcommand{\NumApps}{{\color{black} 10}\xspace}
\newcommand{\NumRulesFixed}{{\color{black} 4}\xspace}
\newcommand{\NumPredicatesRelaxed}{{\color{black} 17}\xspace}
\newcommand{\NumOrderingProblems}{{\color{black} 46}\xspace}
\newcommand{\NumRulesSynthesized}{{\color{black} 2632}\xspace}
\newcommand{\NumOpSequences}{{\color{black} 6246}\xspace}
\newcommand{\NumFailureExamples}{{\color{black} 61000}\xspace}
\newcommand{\NumSimplifiedExpressions}{{\color{black} 195371}\xspace} 
\newcommand{\NumBugsAutomated}{{\color{black} 5}\xspace}
\newcommand{\NumOriginalRules}{{\color{black} 999}\xspace}


%% Copyright information
%% Supplied to authors (based on authors' rights management selection;
%% see authors.acm.org) by publisher for camera-ready submission;
%% use 'none' for review submission.
\setcopyright{none}
%\setcopyright{acmcopyright}
%\setcopyright{acmlicensed}
%\setcopyright{rightsretained}
%\copyrightyear{2018}           %% If different from \acmYear

%% Bibliography style
\bibliographystyle{ACM-Reference-Format}
%% Citation style
%\citestyle{acmauthoryear}  %% For author/year citations
%\citestyle{acmnumeric}     %% For numeric citations
%\setcitestyle{nosort}      %% With 'acmnumeric', to disable automatic
                            %% sorting of references within a single citation;
                            %% e.g., \cite{Smith99,Carpenter05,Baker12}
                            %% rendered as [14,5,2] rather than [2,5,14].
%\setcitesyle{nocompress}   %% With 'acmnumeric', to disable automatic
                            %% compression of sequential references within a
                            %% single citation;
                            %% e.g., \cite{Baker12,Baker14,Baker16}
                            %% rendered as [2,3,4] rather than [2-4].


%%%%%%%%%%%%%%%%%%%%%%%%%%%%%%%%%%%%%%%%%%%%%%%%%%%%%%%%%%%%%%%%%%%%%%
%% Note: Authors migrating a paper from traditional SIGPLAN
%% proceedings format to PACMPL format must update the
%% '\documentclass' and topmatter commands above; see
%% 'acmart-pacmpl-template.tex'.
%%%%%%%%%%%%%%%%%%%%%%%%%%%%%%%%%%%%%%%%%%%%%%%%%%%%%%%%%%%%%%%%%%%%%%


%% Some recommended packages.
\usepackage{booktabs}   %% For formal tables:
                        %% http://ctan.org/pkg/booktabs
\usepackage{subcaption} %% For complex figures with subfigures/subcaptions
                        %% http://ctan.org/pkg/subcaption


\begin{document}

%% Title information
\title{Verifying and Improving Halide’s Term Rewriting System with Program Synthesis}         %% [Short Title] is optional;
                                        %% when present, will be used in
                                        %% header instead of Full Title.
%\titlenote{with title note}             %% \titlenote is optional;
                                        %% can be repeated if necessary;
                                        %% contents suppressed with 'anonymous'
\subtitle{Appendix}                     %% \subtitle is optional
%\subtitlenote{with subtitle note}       %% \subtitlenote is optional;
                                        %% can be repeated if necessary;
                                        %% contents suppressed with 'anonymous'


%% Author information
%% Contents and number of authors suppressed with 'anonymous'.
%% Each author should be introduced by \author, followed by
%% \authornote (optional), \orcid (optional), \affiliation, and
%% \email.
%% An author may have multiple affiliations and/or emails; repeat the
%% appropriate command.
%% Many elements are not rendered, but should be provided for metadata
%% extraction tools.

%% Author with single affiliation.
\author{First1 Last1}
\authornote{with author1 note}          %% \authornote is optional;
                                        %% can be repeated if necessary
\orcid{nnnn-nnnn-nnnn-nnnn}             %% \orcid is optional
\affiliation{
  \position{Position1}
  \department{Department1}              %% \department is recommended
  \institution{Institution1}            %% \institution is required
  \streetaddress{Street1 Address1}
  \city{City1}
  \state{State1}
  \postcode{Post-Code1}
  \country{Country1}                    %% \country is recommended
}
\email{first1.last1@inst1.edu}          %% \email is recommended

%% Author with two affiliations and emails.
\author{First2 Last2}
\authornote{with author2 note}          %% \authornote is optional;
                                        %% can be repeated if necessary
\orcid{nnnn-nnnn-nnnn-nnnn}             %% \orcid is optional
\affiliation{
  \position{Position2a}
  \department{Department2a}             %% \department is recommended
  \institution{Institution2a}           %% \institution is required
  \streetaddress{Street2a Address2a}
  \city{City2a}
  \state{State2a}
  \postcode{Post-Code2a}
  \country{Country2a}                   %% \country is recommended
}
\email{first2.last2@inst2a.com}         %% \email is recommended
\affiliation{
  \position{Position2b}
  \department{Department2b}             %% \department is recommended
  \institution{Institution2b}           %% \institution is required
  \streetaddress{Street3b Address2b}
  \city{City2b}
  \state{State2b}
  \postcode{Post-Code2b}
  \country{Country2b}                   %% \country is recommended
}
\email{first2.last2@inst2b.org}         %% \email is recommended





%% Keywords
%% comma separated list
%\keywords{keyword1, keyword2, keyword3}  %% \keywords are mandatory in final camera-ready submission


%% \maketitle
%% Note: \maketitle command must come after title commands, author
%% commands, abstract environment, Computing Classification System
%% environment and commands, and keywords command.
\maketitle

\appendix
\section{Appendix}

\subsection{Halide expression grammar}
\label{ss:appendixA}


\begin{grammar}
<Expr> ::= <BoolExpr> 
\alt <IntExpr> 
\alt <VectorExpr>

<BoolExpr> ::= `true'
\alt `false'
\alt <IntExpr> `<' <IntExpr>
\alt <IntExpr> `>' <IntExpr>
\alt <IntExpr> `<=' <IntExpr>
\alt <IntExpr> `>=' <IntExpr>
\alt <IntExpr> `=' <IntExpr>
\alt <IntExpr> `!=' <IntExpr>
\alt <VectorExpr> `<' <VectorExpr>
\alt <VectorExpr> `>' <VectorExpr>
\alt <VectorExpr> `<=' <VectorExpr>
\alt <VectorExpr> `>=' <VectorExpr>
\alt <VectorExpr> `=' <VectorExpr>
\alt <VectorExpr> `!=' <VectorExpr>
\alt <BoolExpr> `&&' <BoolExpr>
\alt <BoolExpr> `||' <BoolExpr>
\alt `!' <BoolExpr>

<IntExpr> ::= <IntExpr> `+' <IntExpr>
\alt <IntExpr> `-' <IntExpr>
\alt <IntExpr> `*' <IntExpr>
\alt <IntExpr> `/' <IntExpr>
\alt <IntExpr> `\%' <IntExpr>
\alt `max' <IntExpr> <IntExpr>
\alt `min' <IntExpr> <IntExpr>
\alt `select' <BoolExpr> <IntExpr> <IntExpr>
\alt integers

<VectorExpr> ::= `broadcast' <IntExpr>
\alt `ramp' <IntExpr> <IntExpr>
\alt <VectorExpr> `+' <VectorExpr>
\alt <VectorExpr> `-' <VectorExpr>
\alt <VectorExpr> `*' <VectorExpr>
\alt <VectorExpr> `/' <VectorExpr>
\alt <VectorExpr> `\%' <VectorExpr>
\alt `max' <VectorExpr> <VectorExpr>
\alt `min' <VectorExpr> <VectorExpr>
\alt `select' <BoolExpr> <VectorExpr> <VectorExpr>
\end{grammar}

\subsection{The full Halide reduction order}
\label{a:reductionorder}

\begin{equation}
s >_{\mathcal{V}ar} t \textrm{ iff } \exists x . |s|_{x} > |t|_{x} \wedge \lnot \exists \sigma . \sigma(x) = \emptyset
\end{equation}

The order $>_{\mathcal{V}ar}$ holds if there is one variable with strictly fewer occurrences in $s$ than in $t$, and if that variable cannot be a ground term. 

\begin{equation}
s >_{vec} t \textrm{ iff } |s|_{vec} > |t|_{vec} \wedge \forall x \in \mathcal{V}ar . |s|_x \geq |t|_x
\end{equation}

The order $>_{vec}$ holds if there are strictly more vector operations in $s$ than in $t$.

\begin{equation}
s >_{\textrm{dmm}} t \textrm{ iff } |s|_{\textrm{dmm}} > |t|_{\textrm{dmm}} \wedge \forall x \in V, |s|_x \geq |t|_x
\end{equation}

The order $>_{\textrm{dmm}}$ holds if the total count of division, modulus, and multiplication operations is strictly greater in $s$ than in $t$.

\begin{equation}
s >_{\textrm{leaf}} t \textrm{ iff } |s|_{\textrm{leaf}} > |t|_{\textrm{leaf}} \wedge \forall x \in V, |s|_x \geq |t|_x
\end{equation}

We define the measure function $|s|_{\textrm{leaf}}$ as the number of leaves or terminals in the term $s$ represented as an abstract syntax tree. Thus $>_{\textrm{leaf}}$ holds if $s$ has more leaves than $t$.

\begin{equation}
s >_{\textrm{op}} t \textrm{ iff } \sum_{f \in \Sigma} |s|_f > \sum_{f \in \Sigma} |t|_f \wedge \forall x \in V, |s|_x \geq |t|_x
\end{equation}

The order $>_{\textrm{op}}$ holds if the count of all operations (all symbols in $\Sigma$ that are not zero-arity) in term $s$ is strictly greater that of $t$.

\begin{equation}
s >_{0/1} t \textrm{ iff } |s|_0 + |s|_1 < |t|_0 + |t|_1 \wedge \forall x \in \mathcal{V}ar(s) . \lnot \exists \sigma . (\sigma(x) \neq 0 \wedge \sigma(x) \neq 1)
\end{equation}

The order $>_{0/1}$ holds if there are strictly more occurrences of the constants 0 and 1 in $t$ than in $s$ and if for all variables in $s$, there is no substitution such that $x$ can be substituted with the constants 0 or 1.

\begin{equation}
s >_{f} t \textrm{ iff } |s|_{f} > |t|_{f} \wedge \forall x \in \mathcal{V}ar . |s|_x \geq |t|_x
\end{equation} 

This order represents a composition of orders over each operation in the Halide expression grammar. The operations are checked in this order:

\begin{enumerate}
  \item \texttt{ramp}
  \item \texttt{broadcast}
  \item \texttt{select}
  \item division
  \item multiply
  \item modulus
  \item subtraction
  \item addition
  \item \texttt{min}, \texttt{max}
  \item or
  \item and
  \item $\geq$
  \item $>$
  \item $\leq$
  \item $<$
  \item $\neq$
  \item $=$
  \item not
\end{enumerate}

\begin{equation}
s >_{lpo} t
\end{equation}
The order $>_{lpo}$ is a lexicographic path order induced by an order defined over the Halide operations. We further refine the Halide operations by distinguishing addition, multiplication, select, min, and max where both arguments must not be ground terms from their counterparts whose arguments may be a ground term. We also separate subtraction where the left operand is the constant 0 from substractions where the left operand cannot be the constant 0.

\begin{itemize}
  \item add by constant <
  \item multiply by constant <
  \item \{subtract from 0, select from constant \} < 
  \item max with constant <
  \item \{min with constant, max, min, add, mul, subtract \} <
  \item all other operators
\end{itemize}

\begin{equation}
s >_{str} t \textrm{ iff } \phi_{str}(s) < \phi_{str}(t) \wedge \forall x \in V, |s|_x \geq |t|_x
\end{equation}

The order $>_{str}$ is checked by calculating the function $\phi_{str}$, which traverses the input term depth-first and replaces each constant with "b" and each variable with "a". The resulting strings are then compared lexicographically.

\begin{equation}
s >_{negc} t \textrm{ iff } |s|_{negc} > |t|_{negc}  \wedge \forall x \in V, |s|_x \geq |t|_x
\end{equation}

The order $>_{negc}$ holds when $s$ contains strictly more negative constants than $t$.

\begin{equation}
s >_{nmodc} t \textrm{ iff } |s|_{nmodc} > |t|_{nmodc}  \wedge \forall x \in V, |s|_x \geq |t|_x
\end{equation}

The order $>_{nmodc}$ holds when $s$ contains strictly more constants that are not within the range $0 \leq c < |c_m|$, for some constant $c_m$. 

\begin{equation}
s >_{uniq} t \textrm{ iff } |s|_{uniq} > |t|_{uniq}  \wedge \forall x \in V, |s|_x \geq |t|_x
\end{equation}

The order $>_{uniq}$ holds when $s$ contains strictly more unique subtrees than $t$.

Finally, these two rules are well-ordered because they associate min and max to the left.

\begin{equation}
\tag{max106}
\texttt{max}(\texttt{max}(x,y), \texttt{max}(z,w)) \rightarrow_R \texttt{max}(\texttt{max}(\texttt{max}(x,y),z),w)
\end{equation}

\begin{equation}
\tag{min106}
\texttt{min}(\texttt{min}(x,y), \texttt{min}(z,w)) \rightarrow_R \texttt{min}(\texttt{min}(\texttt{min}(x,y),z),w)
\end{equation}

\end{document}