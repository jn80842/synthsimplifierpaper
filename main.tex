%% For double-blind review submission, w/o CCS and ACM Reference (max submission space)
\documentclass[sigplan,review,anonymous]{acmart}\settopmatter{printfolios=true,printccs=false,printacmref=false}
%% For double-blind review submission, w/ CCS and ACM Reference
%\documentclass[sigplan,review,anonymous]{acmart}\settopmatter{printfolios=true}
%% For single-blind review submission, w/o CCS and ACM Reference (max submission space)
%\documentclass[sigplan,review]{acmart}\settopmatter{printfolios=true,printccs=false,printacmref=false}
%% For single-blind review submission, w/ CCS and ACM Reference
%\documentclass[sigplan,review]{acmart}\settopmatter{printfolios=true}
%% For final camera-ready submission, w/ required CCS and ACM Reference
%\documentclass[sigplan]{acmart}\settopmatter{}
\usepackage{amsmath}
\definecolor{uwpurple}{RGB}{128,0,128}
\newcommand{\jln}[1]{\textcolor{uwpurple}{\textit{[{#1} --JLN]}}}
\usepackage{syntax}
%% Conference information
%% Supplied to authors by publisher for camera-ready submission;
%% use defaults for review submission.
\acmConference[PL'18]{ACM SIGPLAN Conference on Programming Languages}{January 01--03, 2018}{New York, NY, USA}
\acmYear{2018}
\acmISBN{} % \acmISBN{978-x-xxxx-xxxx-x/YY/MM}
\acmDOI{} % \acmDOI{10.1145/nnnnnnn.nnnnnnn}
\startPage{1}

%% Copyright information
%% Supplied to authors (based on authors' rights management selection;
%% see authors.acm.org) by publisher for camera-ready submission;
%% use 'none' for review submission.
\setcopyright{none}
%\setcopyright{acmcopyright}
%\setcopyright{acmlicensed}
%\setcopyright{rightsretained}
%\copyrightyear{2018}           %% If different from \acmYear

%% Bibliography style
\bibliographystyle{ACM-Reference-Format}
%% Citation style
%\citestyle{acmauthoryear}  %% For author/year citations
%\citestyle{acmnumeric}     %% For numeric citations
%\setcitestyle{nosort}      %% With 'acmnumeric', to disable automatic
                            %% sorting of references within a single citation;
                            %% e.g., \cite{Smith99,Carpenter05,Baker12}
                            %% rendered as [14,5,2] rather than [2,5,14].
%\setcitesyle{nocompress}   %% With 'acmnumeric', to disable automatic
                            %% compression of sequential references within a
                            %% single citation;
                            %% e.g., \cite{Baker12,Baker14,Baker16}
                            %% rendered as [2,3,4] rather than [2-4].


%%%%%%%%%%%%%%%%%%%%%%%%%%%%%%%%%%%%%%%%%%%%%%%%%%%%%%%%%%%%%%%%%%%%%%
%% Note: Authors migrating a paper from traditional SIGPLAN
%% proceedings format to PACMPL format must update the
%% '\documentclass' and topmatter commands above; see
%% 'acmart-pacmpl-template.tex'.
%%%%%%%%%%%%%%%%%%%%%%%%%%%%%%%%%%%%%%%%%%%%%%%%%%%%%%%%%%%%%%%%%%%%%%


%% Some recommended packages.
\usepackage{booktabs}   %% For formal tables:
                        %% http://ctan.org/pkg/booktabs
\usepackage{subcaption} %% For complex figures with subfigures/subcaptions
                        %% http://ctan.org/pkg/subcaption


\begin{document}

%% Title information
\title{Automatically improving Halide’s simplifier with synthesis}         %% [Short Title] is optional;
                                        %% when present, will be used in
                                        %% header instead of Full Title.
\titlenote{with title note}             %% \titlenote is optional;
                                        %% can be repeated if necessary;
                                        %% contents suppressed with 'anonymous'
\subtitle{Subtitle}                     %% \subtitle is optional
\subtitlenote{with subtitle note}       %% \subtitlenote is optional;
                                        %% can be repeated if necessary;
                                        %% contents suppressed with 'anonymous'


%% Author information
%% Contents and number of authors suppressed with 'anonymous'.
%% Each author should be introduced by \author, followed by
%% \authornote (optional), \orcid (optional), \affiliation, and
%% \email.
%% An author may have multiple affiliations and/or emails; repeat the
%% appropriate command.
%% Many elements are not rendered, but should be provided for metadata
%% extraction tools.

%% Author with single affiliation.
\author{First1 Last1}
\authornote{with author1 note}          %% \authornote is optional;
                                        %% can be repeated if necessary
\orcid{nnnn-nnnn-nnnn-nnnn}             %% \orcid is optional
\affiliation{
  \position{Position1}
  \department{Department1}              %% \department is recommended
  \institution{Institution1}            %% \institution is required
  \streetaddress{Street1 Address1}
  \city{City1}
  \state{State1}
  \postcode{Post-Code1}
  \country{Country1}                    %% \country is recommended
}
\email{first1.last1@inst1.edu}          %% \email is recommended

%% Author with two affiliations and emails.
\author{First2 Last2}
\authornote{with author2 note}          %% \authornote is optional;
                                        %% can be repeated if necessary
\orcid{nnnn-nnnn-nnnn-nnnn}             %% \orcid is optional
\affiliation{
  \position{Position2a}
  \department{Department2a}             %% \department is recommended
  \institution{Institution2a}           %% \institution is required
  \streetaddress{Street2a Address2a}
  \city{City2a}
  \state{State2a}
  \postcode{Post-Code2a}
  \country{Country2a}                   %% \country is recommended
}
\email{first2.last2@inst2a.com}         %% \email is recommended
\affiliation{
  \position{Position2b}
  \department{Department2b}             %% \department is recommended
  \institution{Institution2b}           %% \institution is required
  \streetaddress{Street3b Address2b}
  \city{City2b}
  \state{State2b}
  \postcode{Post-Code2b}
  \country{Country2b}                   %% \country is recommended
}
\email{first2.last2@inst2b.org}         %% \email is recommended


%% Abstract
%% Note: \begin{abstract}...\end{abstract} environment must come
%% before \maketitle command
\begin{abstract}
Text of abstract \ldots.
\end{abstract}


%% 2012 ACM Computing Classification System (CSS) concepts
%% Generate at 'http://dl.acm.org/ccs/ccs.cfm'.
\begin{CCSXML}
<ccs2012>
<concept>
<concept_id>10011007.10011006.10011008</concept_id>
<concept_desc>Software and its engineering~General programming languages</concept_desc>
<concept_significance>500</concept_significance>
</concept>
<concept>
<concept_id>10003456.10003457.10003521.10003525</concept_id>
<concept_desc>Social and professional topics~History of programming languages</concept_desc>
<concept_significance>300</concept_significance>
</concept>
</ccs2012>
\end{CCSXML}

\ccsdesc[500]{Software and its engineering~General programming languages}
\ccsdesc[300]{Social and professional topics~History of programming languages}
%% End of generated code


%% Keywords
%% comma separated list
\keywords{keyword1, keyword2, keyword3}  %% \keywords are mandatory in final camera-ready submission


%% \maketitle
%% Note: \maketitle command must come after title commands, author
%% commands, abstract environment, Computing Classification System
%% environment and commands, and keywords command.
\maketitle


\section{Introduction}

\subsection{Motivation}

The Halide compiler contains a term rewriting system, currently comprised of several hundred rules, that operates over the space of Halide expressions\footnote{See Appendix ~\ref{ss:appendixA} for the full Halide expression grammar}. This TRS often serves as a proof engine to determine if some equality or inequality holds; for example, in order to parallelize a reduction variable, the compiler must prove that there are no hazards. The compiler also frequently needs to \emph{simplify} Halide expressions. For example, canceling correlated subexpressions aids in proving monotonicity in sliding window optimizations; finding a tight bound on loops or memory allocations is often made possible when expressions have been simplified. \jln{this is handwave-y, fix later} For this reason, we often refer to the Halide term rewriting system as a simplifier.

The design of the term rewriting algorithm has two important requirements. The first is performance: although the simplifier is called at compile time, it may be invoked hundreds of times in compiling a single pipeline \jln{is this true?}. Thus, unlike many term rewriting algorithms, the Halide TRS never backtracks. The second requirement is determinisim: the compiler must always return the same schedule every time it encounters a particular pipeline. This precludes any non-deterministic search strategy for the best sequence of rules to apply, or directly invoking a solver such as Z3, since even with a fixed random seed machines with different computational power may return different results within a given timeout.

\subsection{The Halide TRS algorithm}

The Halide term rewriting algorithm simplifies an input expression in a depth-first, bottom-up traversal of the expression DAG. At each node, it looks up the list of rewrite rules that correspond to the node's function symbol, then attempts to match the subtree expression with the rule LHSs in order. When a match is found, it rewrites the subtree expression using the RHS of that rule, and then attempts to simplify the subtree expression again. If no rule matches the subtree, the traversal continues; when the entire expression cannot be simplified further, the rewritten expression is returned. \jln{use example from slides here; also confirm big O performance of this algo}

Halide rewrite rules optionally contain a predicate that must hold for a rule to be applied. Halide expressions may contain constants whose values are known at compile time. Predicate expressions may contain only such constants; when the LHS of a rule matches an expression, its predicate is evaluated and only if it is true will the rewrite be applied. \jln{maybe cite a common case where we know a constant, like a loop bound?}

\subsection{Term rewriting systems background}

Terms are defined inductively over a set of variables $V$ and a set of function symbols $\Sigma$. Every variable $v \in V$ is a term, and for any function symbol $f \in \Sigma$ with arity $n$ and any terms $t_1, ..., t_n$, the application of the symbol to the terms $f(t_1, ..., t_n)$ is also a term. We refer to the set of terms constructed from the variables $V$ and the function symbols $\Sigma$ as $T(\Sigma, V)$.

A \emph{rule} is a directed binary relation $l \rightarrow r$ such that $l$ is not a variable, and all variables present in $r$ are also present in $l$ (i.e., $\mathcal{V}ar(l) \supseteq \mathcal{V}ar(r)$). A set of rewrite rules is a \emph{term rewriting system}.

Consider a set of terms $T(\Sigma, V)$ such that $\Sigma = \{\otimes, \oplus\}$ and $V$ be an infinite set of variables. Let the term rewriting system $R$ consist of a single rule:

\[ R = \{ x_1 \otimes x_2 \rightarrow x_1 \oplus x_2 \} \]

We use $R$ to rewrite the term

\[ 
(y_1 \oplus y_1) \otimes (y_2 \otimes y_3)
\]

The first step is matching; we find a substitution that will unify the lefthand side (LHS) of the rule with the term we are rewriting. Here, the substitution is:

\[
\{ x_1 \mapsto (y_1 \oplus y_1), x_2 \mapsto (y_2 \otimes y_3) \}
\]

We then apply this substitution to the righthand side (RHS) of the rule to obtain the rewritten version of the original term.

\[ 
(y_1 \oplus y_1) \oplus (y_2 \otimes y_3)
\]

\subsection{Limitations}

We limit ourselves to expressions over the set of infinite precision integers in this work, although Halide expressions can contain floats and fixed-width integers.

\jln{We haven't synthesized any vector op rules; not sure if that's a conscious decision or we haven't really had any in input expressions.}

\section{Soundness}

\subsection{Rule verification}

First, we assure soundness by verifying the current set of handwritten rules. We do this by modeling Halide expressions in SMT2 and sending them to the SMT solver Z3~\cite{de2008z3}. Most of the Halide expression semantics maps cleanly onto SMT2 formulas. The functions \texttt{max} and \texttt{min} are defined in the usual way, and \texttt{select} has the same meaning as the SMT2 operator \texttt{ite}. Division and modulo are given the Euclidean definitions in both Halide and SMT2. If a variable appears in the LHS of a rule as a divisor in a division or modulo operation, it is assumed to be non-zero. The Halide expressions do not have a true boolean type (true and false are represented by unsigned integers of 1 bitwidth), so expressions must be typed as either \texttt{Int} or \texttt{Bool} when translated into SMT2. The Halide expression grammar contains two vector operators, \texttt{broadcast} and \texttt{ramp}; all other integer operators can be coerced to vector operators. The operation \texttt{broadcast} projects some value $x$ to a vector of length $l$; because of the type coercion, we can simply represent \texttt{broadcast(x)} as the variable \texttt{x} in SMT2. The \texttt{ramp} operator creates a vector of length $l$ whose initial entry has the value $x$ and all subsequent entries increase with stride $s$. In SMT2, we represent this term as the symbolic expression $x + l * s$, where $l$ must be zero or positive.

Given this modeling, for each rule, we assert any assumptions are true, then assert that the rule's LHS is not equivalent to its RHS and ask Z3 to find a counterexample. If no counterexample can be found, the LHS must be equivalent to the RHS and the rule must be correct. We implemented a SMT2 printer for the Halide rewrite rules and ran them through Z3; out of \jln{TK} rules, we were able to verify \jln{TK} as correct and discover \jln{TK} incorrect rules; these rules are listed in Table~\ref{tab:incorrectrules}.

However, nonlinear integer arithmetic is generally undecidable, and for \jln{TK} rules, Z3 either timed out or returned unknown. Nearly all of these rules used either division or modulo. We used the proof assistant Coq to manually prove or disprove the correctness of these remaining rules. 

\begin{table*}

\caption{Incorrect rules found during verification.}
\begin{tabular}{|l|l|}
\hline
Incorrect rule & Tool used \\
\hline
$((x + c0)/c1)*c1 - x \rightarrow_R x \bmod c1 \textrm{ if } c1 > 0 \wedge c0 + 1 == c1$ & Z3 \\
$x - ((x + c0)/c1)*c1 \rightarrow_R -(x \bmod c1) \textrm{ if } c1 > 0 \wedge c0 + 1 == c1$ & Z3 \\
$min(x, c0) < min(x, c1) + c2 \rightarrow_R \textrm{false if } c0 >= c1 + c2)$ & Z3 \\
$max(x, c0) < max(x, c1) + c2 \rightarrow_R \textrm{false if } c0 >= c1 + c2$ & Z3 \\
\hline
\end{tabular}
\label{tab:incorrectrules}
\end{table*}

\subsection{Termination}

\jln{note that the ``strength'' of operators is arbitrary for the purposes of proving termination, but meaningful in terms of simplifying (rather than proving) expressions.}

Term rewriting systems are not guaranteed to terminate. Consider a term rewriting system containing only one rule: $x + y \rightarrow y + x$. The term $3 + 5$ matches the LHS of the rule and is rewritten to $5 + 3$, which can again be matched to the rule and rewritten to $3 + 5$, and so on. To guarantee termination irrespective of the rule application algorithm, every application of a rule needs to strictly (monotonically) decrease in some measure. In other words, if $s \rightarrow_R t$, $s > t$ for some strict order $>$. However, since the set of terms is infinite, we cannot check that $s > t$ for all pairs of terms $s, t$. Instead, we show that for every rule $l \rightarrow r$ in our term rewriting system, $l > r$. This will suffice if $>$ is a \emph{reduction order}, given:

\begin{theorem}\label{theorem:terminates}
A term rewriting system $R$ terminates iff there exists a reduction order $>$ that satisfies $l > r$ for all $l \rightarrow r \in R$.
\end{theorem}

Per Baader and Nipkow~\cite{baader1999term}, a reduction order is defined as follows.

\begin{definition}
A strict order on terms $T(\Sigma, V)$ is a reduction order iff: 
\begin{enumerate}
    \item well-founded, or terminating,
    \item compatible with $\Sigma$-operations: for all $s_1, s_2 \in T(\Sigma,V)$, all $n \geq 0$, and all $f \in \Sigma^{(n)}$, $s_1 > s_2$ implies
    \[ f(t_1,...t_{i-1},s_1,t_{i+1},...,t_n) > f(t_1,...t_{i-1},s_2,t_{i+1},...,t_n)
    \]
    for all $i, 1 \leq i \leq n$ and all $t_1,...t_{i-1},t_{i+1},...,t_n \in T(\Sigma,V)$.
    \item closed under substitution: for all $s_1, s_2 \in T(\Sigma,V)$ and all substitutions $\sigma \in \mathcal{S}ub(T(\Sigma,V))$, $s_1 > s_2$ implies $\sigma(s_1) > \sigma(s_2)$.
\end{enumerate}
\end{definition}

We demonstrate our reduction order over a simpler set of terms defined by the function symbol set $\Sigma = \{\oplus, \otimes\}$ and the unbounded set of variables $V = \{x_1, x_2, \ldots \}$. This easily generalizes to the full set of Halide terms.

First, we need to define an order that is well-founded. We do this by defining a measure function that embeds terms into the natural numbers. We define a function $\phi_\otimes$:

\[
\phi_\otimes(s) = \textrm{count of occurrences of } \otimes \textrm{ in the term } s
\]

We can use this measure function to define a strict order $>_\otimes$:

\[
s_1 >_\otimes s_2 \iff \phi_\otimes(s_1) > \phi_\otimes(s_2)
\]

Since $\phi_\otimes$ provides a mapping from terms to the natural numbers, $<_\otimes$ must be well-founded. We can also easily see that $>_\otimes$ is compatible with $\Sigma$-operations.

\[
\begin{aligned}
\phi_\otimes(f(t_1, \ldots, t_{i-1}, s_1, t_{i+1}, \ldots, t_n)) = \\
\phi_\otimes(s_1) + \sum^{t_i} \phi_\otimes(t_i) + \begin{cases} 1 & f = \otimes \\
0 & \textrm{otherwise} \end{cases}
\end{aligned}
\]

Since the summation of the $\phi_\otimes(t_i)$ terms and the piecewise functions are the same for both terms, the property holds.

However, the order is not closed under substitution. Consider:

\[
x_1 \otimes x_2 >_\otimes x_2 \oplus x_2
\]
\[
\sigma(x_2) = x_3 \otimes x_3
\]

Applying the substitution produces more $\otimes$ operations in the righthand term than on the left.

\[
x_1 \otimes (x_3 \otimes x_3) \ngtr_\otimes (x_3 \otimes x_3) \oplus (x_3 \otimes x_3)
\]
 
We must refine the definition of our order. We define a new function $|s|_x$ to denote the number of occurrences of the variable $x$ in the term $s$. Thus, $|x_1 \otimes x_2|_{x_1} = 1$ and $|x_1 \oplus x_1|_{x_1} = 2$. We redefine the order $>_\otimes$ to add the condition that for all variables $x \in V$, the number of occurrences of the variable $x$ in $s_1$ must be equal or greater than the number of occurrences in $s_2$.
\[
s_1 >_\otimes s_2 \iff \phi_\otimes(s_1) > \phi_\otimes(s_2) \wedge \forall x \in V, |s_1|_x \geq |s_2|_x
\]

The redefined order is still well-founded and compatible with $\Sigma$-operations. To show that our redefined order is closed under substitutions via proof by contradiction, let us assume that for some $s, t$, $s >_\otimes t$ and for some substitution $\sigma$, $\sigma(s) \ngtr_\otimes \sigma(t)$. Substitutions can only cause the value of $\phi_\otimes$ to increase. Let $\sigma = \{x_1 \mapsto t_1\}$, and let $s$ be a term that contains the variable $x_1$. If $\phi_\otimes(t_1) = 0$, then $\phi_\otimes(\sigma(s)) = \phi_\otimes(s)$. If $\phi_\otimes(t_1) > 0$, then $\phi_\otimes(\sigma(s)) > \phi_\otimes(s)$. Thus, in order to find some $\sigma$ such that $s >_\otimes t \wedge \sigma(s) \ngtr_\otimes \sigma(t)$, we need to find some $x$ in the term $t$ that increases the count of $\otimes$ operations. But, any such $x$ must occur in $t$ an equal or greater number of times in $s$ and thus must increase the $\phi_\otimes(\sigma(s))$ by an equal or greater value than the increase from $\phi_\otimes(t)$ to $\phi_\otimes(\sigma(t))$. Thus, a contradiction is found and the assertion is proved.


\begin{theorem}
The lexicographic product of two terminating relations is again terminating.
\end{theorem}

Thus, we define an order using a measure function for every $f \in \Sigma$, and then take the lexicographic product of these orders to define a reduction order. Since we evaluate this order by calculating a histogram of operators in each term, we refer to this order as $<_H$. The order in which the function counts are considered lexicographically is given in Appendix~\ref{symbolstrength}. 

Note that while variable elimination is desirable, we cannot use a measure function that counts the number of variable occurrences in a term when composed with the other orders. For example, let $\Sigma = \{\oplus, \otimes\}$ and $V = \{x, y\}$, and let $>_H$ be the lexicographic product of the orders $>_{\oplus}, >_{\otimes}, >_v$. Here $x \otimes x >_H x$, but the order is not closed under substitution. For a substitution $\sigma(x) = y \oplus y$, we have $(y \oplus y) \otimes (y \oplus y) \ngtr y \oplus y$. 

However, we frequently want to normalize terms without changing the number of occurrences of function symbols within them. To permit such rules, we take the lexicographic order of $>_H$ with one more order. The measure function for our last tie-breaking order takes the root symbol of the AST of a term and translate it to a number based on the function count priority as given above. Since we prefer ``stronger'' symbols to be higher in the tree, here the order is reversed. For example, the root symbol of $select(x, y, z) - y$ is $-$ and the root symbol of $select(x, 0, z - y))$; since $-$ is ``weaker'' than $select$, we say $select(x, y, z) - y >_{root} select(x, 0, z - y))$.

\jln{Here we need to either talk about existimg rules that don't conform to our reduction order, or figure out how to deal with those rules}



\section{Completeness}

The Halide term rewriting system is not complete, by which we mean that there exist expressions checked by the compiler running on real-world pipelines that the simplifier cannot resolve, but which are known to be true. \jln{there should be more discussion of TRS properties here (confluence etc) and whether they are decidable for the Halide lang} We propose a workflow to augment the simplifier using synthesis.

\section{Related Work}

Program synthesis has been applied to term rewriting systems in Swapper~\cite{singh2016swapper}, a tool that learns a set of rewrite formulas to transform SMT formulas into forms that can be more easily solved by theory solvers. Swapper takes a training set of formulas drawn from a target domain, does probabilistic sampling to choose LHS candidates, synthesizes predicates and RHS, and finally autotunes the resulting ruleset to choose a rule subset and rule order that gives the best solver performance. Unlike our work, Swapper is an optimizer rather than a prover. Because our rewrite algorithm scales well in terms of the number of rules, we simply use all candidate LHS terms for synthesis rather than sampling. Swapper synthesizes RHS terms that have fewer AST nodes than their corresponding LHS terms; this is not a reduction order and does not guarantee termination, although the autotuning step should reject rulesets with cycles. Finally, Swapper creates a fresh term rewriting system with each invocation, whereas our work augments and maintains an existing term rewriting system.

Butler et al.~\cite{butler2017synthesizing} uses synthesis to learn human-interpretable strategies for puzzle games such as Sudoku or Nonograms. Strategies consist of a pattern (LHS term with variables), condition (predicate), and action (RHS term expressing rewrite), and so comprise a term rewriting system. Strategies are synthesizing Rosette, using a CEGIS loop similar to our synthesis process. Progress in these puzzle games is always monotonic, which gives a natural reduction order. Their goal is to model human strategies, which are greedy and non-deterministic, so they do not consider rule priority or any particular term rewriting algorithm. 

\section{Conclusion}


%% Acknowledgments
\begin{acks}                            %% acks environment is optional
                                        %% contents suppressed with 'anonymous'
  %% Commands \grantsponsor{<sponsorID>}{<name>}{<url>} and
  %% \grantnum[<url>]{<sponsorID>}{<number>} should be used to
  %% acknowledge financial support and will be used by metadata
  %% extraction tools.
  This material is based upon work supported by the
  \grantsponsor{GS100000001}{National Science
    Foundation}{http://dx.doi.org/10.13039/100000001} under Grant
  No.~\grantnum{GS100000001}{nnnnnnn} and Grant
  No.~\grantnum{GS100000001}{mmmmmmm}.  Any opinions, findings, and
  conclusions or recommendations expressed in this material are those
  of the author and do not necessarily reflect the views of the
  National Science Foundation.
\end{acks}


%% Bibliography
\bibliography{bib}


%% Appendix
\appendix
\section{Appendix}

\subsection{Halide expression grammar}
\label{ss:appendixA}

\jln{vector ops are missing}

\begin{grammar}
<Expr> ::= <BoolExpr> 
\alt <IntExpr> 
\alt <VectorExpr>

<BoolExpr> ::= `true'
\alt `false'
\alt <IntExpr> `<' <IntExpr>
\alt <IntExpr> `>' <IntExpr>
\alt <IntExpr> `<=' <IntExpr>
\alt <IntExpr> `>=' <IntExpr>
\alt <IntExpr> `=' <IntExpr>
\alt <IntExpr> `!=' <IntExpr>
\alt <BoolExpr> `&&' <BoolExpr>
\alt <BoolExpr> `||' <BoolExpr>
\alt `!' <BoolExpr>

<IntExpr> ::= <IntExpr> `+' <IntExpr>
\alt <IntExpr> `-' <IntExpr>
\alt <IntExpr> `*' <IntExpr>
\alt <IntExpr> `/' <IntExpr>
\alt <IntExpr> `\%' <IntExpr>
\alt `max' <IntExpr> <IntExpr>
\alt `min' <IntExpr> <IntExpr>
\alt `select' <BoolExpr> <IntExpr> <IntExpr>
\alt integers

<VectorExpr> ::= vectors
\end{grammar}

\subsection{Halide reduction orders}

For some term $s \in T(\Sigma, V)$ and some variable $x \in V$, let $|s|_x$ be the number occurrences of the variable $x$ in the term $s$. For each function symbol $f$ in the set of Halide operators:

\[
\phi_f(s) = \textrm{count of occurrences of } f \textrm{ in the term } s
\] 
\[
s_1 >_f s_2 \iff \phi_f(s_1) > \phi_f(s_2) \wedge \forall x \in V, |s_1|_x \geq |s_2|_x
\]

Let $>_\Sigma$ be the lexicographic product of each order $>_f$, in the order given in the symbol strength list in Appendix~\ref{symbolstrength}. Thus,

\[
\begin{aligned}
s_1 >_\Sigma s_2 \iff \forall x \in V, |s_1|_x \geq |s_2|_x \wedge \\ (\phi_{\texttt{ramp}}(s_1) > \phi_{\texttt{ramp}}(s_2)) \vee \\ (\phi_{\texttt{ramp}}(s_1) = \phi_{\texttt{ramp}}(s_2) \wedge \\ \phi_{\texttt{broadcast}}(s_1) > \phi_{\texttt{broadcast}}(s_1) \vee \ldots
\end{aligned}
\]

Let $\phi_{root}(s)$ be a function that returns the strength of the root symbol of the term $s$, in the reverse order given below. For example, $\phi_{root}(\texttt{x * y}) > \phi_{root}(\texttt{x / y})$. Then,

\[
s_1 <_{root} s_2 \iff \phi_{root}(s_1) < \phi_{root}(s_2)
\]

\subsection{Halide symbol strength} \label{symbolstrength}

The function symbols in the Halide expression signature, in order by strength, greatest to least.

\begin{enumerate}
  \item \texttt{ramp}
  \item \texttt{broadcast}
  \item \texttt{select}
  \item \texttt{/}
  \item \texttt{*}
  \item \texttt{$\%$}
  \item \texttt{-}
  \item \texttt{+}
  \item \texttt{max, min} (equal strength)
  \item \texttt{!}
  \item \texttt{||}
  \item \texttt{$\&\&$}
  \item \texttt{$>=$}
  \item \texttt{$>$}
  \item \texttt{$<=$}
  \item \texttt{$<$}
  \item \texttt{$!=$}
  \item \texttt{=}
\end{enumerate}

\end{document}
